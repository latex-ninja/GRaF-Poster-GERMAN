\definecolor{Text}{RGB}{0.36, 0.54, 0.66}
\definecolor{Element}{RGB}{0.53, 0.66, 0.42}
\definecolor{Attr}{RGB}{0.6, 0.4, 0.8}

\usepackage{listings}
\lstset{language=XML,
showspaces=false,
showtabs=false,
basicstyle=\ttfamily,
columns=fullflexible,
breaklines=true,
showstringspaces=false,
breakatwhitespace=true,
escapeinside={(*@}{@*)},
basicstyle=\ttfamily
}
\lstdefinelanguage{XML}
{
  morestring=[b]",
  morestring=[s]{>}{<},
  morecomment=[s]{<?}{?>},
  stringstyle=\color{Text},
  identifierstyle=\color{Element}\upshape,
  keywordstyle=\color{Attr}\bfseries,
  morekeywords={xmlns,version,type}% list your attributes here
}


\begin{lstinputlisting}% Start your code-block
<TEI>
<text>
<div canonicalRef="Phaedr.-:1,5" type="text">
<head>Phaedr. 1,5</head>
<div type="originaltext">
<head>
Vacca et
<hi type="Vokabelangabe">
capella
<note type="Vokabelangabe">capella,-ae f.: Ziege</note>
</hi>
, ouis et leo.
</head>
<div type="Abschnitt/Sinneinheit">
<head type="Gliederung">Promythion (1-2)</head>
<lg xml:lang="el">
<l metr="u-|u-|u//-|u/-|u-|ux" n="1" real="#senar">
1 Numquam est fidelis cum potente
<hi type="Sacherklärung">
societas
<note type="Sacherklärung">
societas,-atis f.: Als societas im römischen Rechtssystem verstand man eine Bündnisgemeinschaft, deren Gewinne und Verluste, falls nicht anders vereinbart, gleichmäßig auf alle Beteiligten Parteien aufgeteilt wurden. Bedingungen der Verteilung des Gewinns gemäß dem persönlichen Einsatz oder gemäß den Ressourcen, die ein Einzelner in die Gemeinschaft einbringt, werden vor Eintritt der socii in die societas beschlossen (vgl.
<ref target="#Iustin.-inst.:3,25,2">
Iustin. inst. 3,25,2
</ref>
).
</note>
</hi>
</l>
\end{lstinputlisting}